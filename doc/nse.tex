%\documentclass[review,leqno]{siamart1116}
\documentclass[onefignum,onetabnum]{siamart190516} 
\usepackage{amssymb}
\usepackage{graphicx} 
\usepackage{xspace}
\usepackage{multirow		}
\usepackage{bbm}
\usepackage{Macros/mydef}
\usepackage{Macros/remarks}
\usepackage{algorithm}
\usepackage{algpseudocode}
\usepackage[numbers]{natbib}
\usepackage{graphicx}
\usepackage{subcaption}
\usepackage{enumitem}
\usepackage{bm}
%\usepackage{float}
\newsiamthm{example}{Example}
\usepackage{multirow}
\usepackage{adjustbox}

\let\cite=\citet
%====================To be removed============================
\usepackage[notref,notcite]{showkeys} % To see crossreferences.
\newcommand{\murtazo}[1]{\textcolor{blue}{#1}}
\newcommand{\mn}[1]{{\leavevmode\color{red}#1}}
\usepackage[colorinlistoftodos]{todonotes}
%====================To be removed============================

\title{Equal-order finite element approximation of the Navier-Stokes equations using least-squares stabilization}

%\author{Authors}


\begin{document}

\maketitle


\renewcommand{\thefootnote}{\fnsymbol{footnote}} 
\renewcommand{\thefootnote}{\arabic{footnote}}

\tableofcontents

%\begin{abstract}
%\end{abstract}
%
%\begin{keywords}
%  NSE, G2, Least-Squares
%\end{keywords}
%
%\begin{AMS}
%65M60
%\end{AMS}

\pagestyle{myheadings}
\thispagestyle{plain}
\markboth{AUTHORS}{XXXX}

\section{Navier--Stokes equations: strong form}


\begin{equation}
  \begin{aligned}
    \partial_t \bu  + \bu \SCAL \GRAD \bu 
    - \nabla \cdot \sigma &=0, \\
    \hspace{2.2em} 
    \DIV \bu&= 0 
  \end{aligned}
\end{equation}
with $\sigma = -p \bm I + 2 \nu \varepsilon(\bu)$ and
$\varepsilon(\bu)=\frac{1}{2}(\nabla \bu + (\nabla \bu)^\top)$.

\begin{align*} 
\rightarrow \boxed{f_{\text{NS}}(p,\bu)= 0}
\end{align*}

\section{Navier--Stokes equations: weak form}


\begin{equation}
  \begin{aligned}
    \left( \partial_t \bu ,\  \bv \right) 
    + \big( \bu \SCAL \GRAD \bu,\ \bv \big)
    - \big( p,\ \DIV \bv \big) + 
    \big(2 \nu \varepsilon(\bu),\ \varepsilon(\bv) \big)&=0, \\
    \big( \DIV \bu,\ q)&  = 0 
  \end{aligned}
\end{equation}
\begin{align*} 
\rightarrow \boxed{F_{\text{NS}}(p,\bu)= 0}
\end{align*}

%\newpage

\section{GLS NS: extra terms}
\begin{align*}
\boxed{F_{\text{NS}}(p,\bu) + F_{\text{STAB}}(p,\bu) = 0}
\end{align*}
with
\begin{equation}
\begin{aligned}
F_{\text{STAB}} &= \int_\Omega f_{\text{NS}}(p,\bu) \cdot f_{\text{NS}}(q,\bv)
\\&=\int_\Omega
\left(
\begin{array}{c}
    \partial_t \bu  + \bu \SCAL \GRAD \bu 
    + \nabla p - 2 \nu \Delta \bm u \\
    \DIV \bu 
\end{array}
\right)
\cdot
\left(
\begin{array}{c}
    \partial_t \bv  + \bu \SCAL \GRAD \bv 
    + \nabla q - 2 \nu \Delta \bv \\
    \DIV \bv
\end{array}
\right)
\\&=
\left(
\begin{array}{ccc}
    (\partial_t \bu  + \bu \SCAL \GRAD \bu 
    + \nabla p - 2 \nu \Delta \bm u ,\,\partial_t \bv  + \bu \SCAL \GRAD \bv 
    - 2 \nu \Delta \bv) &+& (\nabla \cdot \bu,\, \nabla \cdot \bv) \\
    (\partial_t \bu  + \bu \SCAL \GRAD \bu 
    + \nabla p - 2 \nu \Delta \bm u,\, \nabla q)
\end{array}
\right)
  \end{aligned}
\end{equation}
(1) Ignoring time dependency of test function:
\begin{equation}
  \begin{aligned}
\left(
\begin{array}{ccc}
    (\partial_t \bu  + \bu \SCAL \GRAD \bu 
    + \nabla p - 2 \nu \Delta \bm u ,\,\bu \SCAL \GRAD \bv 
    - 2 \nu \Delta \bv) &+ & (\nabla \cdot \bu,\, \nabla \cdot \bv) \\
    (\partial_t \bu  + \bu \SCAL \GRAD \bu 
    + \nabla p - 2 \nu \Delta \bm u,\, \nabla q)
\end{array}
\right)
  \end{aligned}
\end{equation}
(2) Ignoring second derivatives of test function:
\begin{equation}
  \begin{aligned}
\left(
\begin{array}{ccc}
    \overbrace{(\partial_t \bu  + \bu \SCAL \GRAD \bu 
    + \nabla p  - 2 \nu \Delta \bm u ,\,\bu \SCAL \GRAD \bv)}^{\text{SUPG}} &+ & 
    \overbrace{(\nabla \cdot \bu,\, \nabla \cdot \bv)}^{\text{GD}} \\
    \underbrace{(\partial_t \bu  + \bu \SCAL \GRAD \bu 
    + \nabla p - 2 \nu \Delta \bm u,\, \nabla q)}_{\text{PSPG}} &&
\end{array}
\right)
  \end{aligned}
\end{equation}
%(3a) Ignoring GD:
%\begin{equation}
%  \begin{aligned}
%\left(
%\begin{array}{c}
%    (\partial_t \bu  + \bu \SCAL \GRAD \bu 
%    + \nabla p - 2 \nu \Delta \bm u,\,\bu \SCAL \GRAD \bv) \\
%    (\partial_t \bu  + \bu \SCAL \GRAD \bu 
%    + \nabla p,\, \nabla q)
%\end{array}
%\right)
%  \end{aligned}
%\end{equation}
%(3b) Ignoring second derivatives of ansatz function:
%\begin{equation}
%  \begin{aligned}
%\left(
%\begin{array}{ccc}
%    (\partial_t \bu  + \bu \SCAL \GRAD \bu 
%    + \nabla p,\,\bu \SCAL \GRAD \bv) &+ & 
%    (\nabla \cdot \bu,\, \nabla \cdot \bv) \\
%    (\partial_t \bu  + \bu \SCAL \GRAD \bu 
%    + \nabla p,\, \nabla q) &&
%\end{array}
%\right)
%  \end{aligned}
%\end{equation}
%Q: How to choose $\delta_{\text{SUPG}}$, $\delta_{\text{PSPG}}$, $\delta_{\text{GD}}$?


%\newpage


\section{GLS NS: residual formulation}

\begin{equation}
  \begin{aligned}
    \left( \partial_t \bu ,\  \bv \right) 
    &+ \big( \bu \SCAL \GRAD \bu,\ \bv \big)
    - \big( p,\ \DIV \bv \big) + 
    \big( 2 \nu \varepsilon(\bu),\ \varepsilon(\bv) \big) \\
    & 
    +  
    \underbrace{\delta_1 \big( \partial_t \bu + \bu \SCAL \GRAD \bu + \GRAD p - 2 \nu \Delta \bu,\ \bu \cdot \GRAD \bv \big)}_{\text{SUPG}}  +
    {\underbrace{\delta_2 \big(\DIV \bu, \, \DIV \bv \big)}_{\text{GD}}}
    =0, \\
    \hspace{2.2em} 
    \big( \DIV \bu,\ q)&
    + \underbrace{\delta_1 \big( \partial_t \bu + \bu \SCAL \GRAD \bu + \GRAD p - 2 \nu \Delta \bu,\ \GRAD q \big)}_{\text{PSPG}} = 0 
  \end{aligned}
\end{equation}

\section{Fixed-point system}

\begin{equation}
  \begin{aligned}
    \left( \partial_t \bu ,\  \bv \right) 
    &+ \big( {\color{black}\bu^*} \SCAL \GRAD \bu,\ \bv \big)
    - \big( {\color{black}p^{n+1}},\ \DIV \bv \big) + 
    \big( 2 \nu \varepsilon(\bu),\ \varepsilon(\bv) \big) \\
    & 
    +  
    \underbrace{\delta_1 \big( \partial_t \bu + {\color{black}\bu^*} \SCAL \GRAD \bu + \GRAD p - 2 \nu \Delta \bu,\ {\color{black}\bu^*} \cdot \GRAD \bv \big)}_{\text{SUPG}}  +
    {\underbrace{\delta_2 \big(\DIV \bu, \, \DIV \bv \big)}_{\text{GD}}}
    =0, \\
    \hspace{2.2em} 
    \big( \DIV \bu,\ q)&
    + \underbrace{\delta_1 \big( \partial_t \bu + {\color{black}\bu^*} \SCAL \GRAD \bu + {\color{black}\GRAD p^{n+1}} - 2 \nu \Delta \bu,\ \GRAD q \big)}_{\text{PSPG}} = 0 
  \end{aligned}
\end{equation}

\section{Linearized system}

\begin{equation}
  \begin{aligned}
    \left( {\color{black}\partial_t'} \bu ,\  \bv \right) 
    &+ \big( \bu^* \SCAL \GRAD \bu,\ \bv \big)
    + {\color{black}\big( \bu \SCAL \GRAD \bu^*,\ \bv \big)}
    - \big( p,\ \DIV \bv \big) + 
    \big( 2 \nu \varepsilon(\bu),\ \varepsilon(\bv) \big) \\
    & 
    +  
    \underbrace{\delta_1 \big( {\color{black}\partial_t'} \bu + {\color{black}\bu^*} \SCAL \GRAD \bu +  {\color{black}\bu \SCAL \GRAD \bu^*} + \GRAD p - 2 \nu \Delta \bu,\ {\color{black}\bu^*} \cdot \GRAD \bv \big)}_{\text{SUPG}} 
    \\
    & 
    +  
    \underbrace{{\color{black}\delta_1 \big(\partial_t' \bu^* + \bu^* \SCAL \GRAD \bu^* + \GRAD p^* - 2 \nu \Delta \bu^*,\ \bu \cdot \GRAD \bv \big)}}_{\text{SUPG}}  +
    {\underbrace{\delta_2 \big(\DIV \bu, \, \DIV \bv \big)}_{\text{GD}}}
    =0, \\
    \hspace{2.2em} 
    \big( \DIV \bu,\ q)&
    + \underbrace{\delta_1 \big( {\color{black}\partial_t'} \bu + {\color{black}\bu^*} \SCAL \GRAD \bu + {\color{black}\bu \SCAL \GRAD \bu^*} + \GRAD p - 2 \nu \Delta \bu,\ \GRAD q \big)}_{\text{PSPG}} = 0 
  \end{aligned}
\end{equation}

\section{Time discretization}

\begin{align*}
\partial_t u &=  \alpha_{n+1} u^{n+1} + \alpha_{n} u^{n} + \alpha_{n-1} u^{n-1} + ... \\
\partial_t' u &=  \alpha_{n+1} u^{n+1} 
\end{align*}

%\section{TODO}
%
%\begin{itemize}
%\item shift time-discretization scaling
%\item add time derivative to SUPG/PSPG
%\item add Hessian $\rightarrow$ $2 \nu \Delta \bu$
%\item definition of $\delta_1$ and  $\delta_2$
%\item BB: why is GD missing?
%\end{itemize}

%\newpage

\section{One-step-$\theta$ time discretization}

\begin{equation}
  \begin{aligned}
    \left( \tilde{\bu} ,\  \bv \right) 
    &+ \tau \big( \bu^* \SCAL \GRAD \overline{\bu},\ \bv \big)
    - \tau \big( p^{n+1},\ \DIV \bv \big) + 
    \tau \big( 2 \nu \varepsilon(\overline{\bu}),\ \varepsilon(\bv) \big) \\
    & 
    +  
    \underbrace{\delta_1 \tau \big(\tilde{\bu}/\tau +  \bu^* \SCAL \GRAD \overline{\bu} + \GRAD \overline{p} - 2 \nu \Delta \overline{\bu}, \
    \bu^* \SCAL \GRAD \bv\big)}_{\text{SUPG}}  +
    \underbrace{\delta_2 \tau \big(\DIV \overline{\bu}, \, \DIV \bv \big)}_{\text{GD}}
    =0, \\
    \hspace{2.2em} 
    \big( \DIV \overline{\bu},\ q)&
    + \underbrace{\delta_1 \big( \tilde{\bu}/\tau +  \bu^* \SCAL \GRAD \overline{\bu} + \GRAD p^{n+1}  - 2 \nu \Delta \overline{\bu},\ \GRAD q \big)}_{\text{PSPG}} = 0 
  \end{aligned}
\end{equation}

\subsection{Fixed-point iteration}

Introduce fixed-point iteration around $\bu^*$ and
discretize with Crank-Nicolson:
\begin{equation}
  \begin{aligned}
    \left( \tilde{\bu} ,\  \bv \right) 
    &+ \tau \big( \bu^* \SCAL \GRAD \overline{\bu},\ \bv \big)
    - \tau \big( p^{n+1},\ \DIV \bv \big) + 
    \tau \big( 2 \nu \varepsilon(\overline{\bu}),\ \varepsilon(\bv) \big) \\
    & 
    +  
    \underbrace{\delta_1 \tau \big( \bu^* \SCAL \GRAD \overline{\bu} + \GRAD \overline{p}, \
    \bu^* \SCAL \GRAD \bv\big)}_{\text{SUPG}}  +
    \underbrace{\delta_2 \tau \big(\DIV \overline{\bu}, \, \DIV \bv \big)}_{\text{GD}}
    =0, \\
    \hspace{2.2em} 
    \big( \DIV \overline{\bu},\ q)&
    + \underbrace{\delta_1 \big( \bu^* \SCAL \GRAD \overline{\bu} + \GRAD p^{n+1} ,\ \GRAD q \big)}_{\text{PSPG}} = 0 
  \end{aligned}
\end{equation}
with
$\overline{p} := \theta p^{n+1} + (1-\theta) p^{n}$,
$\tilde{\bu} := \bu^{n+1} - \bu^{n}$,
$\overline{\bu} := \theta \bu^{n+1} + (1-\theta) \bu^{n}$, 
and $\bu^*_i:=\bu^n$.

%\vspace{3cm}
%
%\section{Lethe}
%
%\begin{equation}
%  \begin{aligned}
%    \left( \partial_t \bu ,\  \bv \right) 
%    &+ \big( \bu \SCAL \GRAD \bu,\ \bv \big)
%    - \big( p,\ \DIV \bv \big) + 
%    \big( 2 \nu \varepsilon(\bu),\ \varepsilon(\bv) \big) \\
%    & 
%    +  
%    \underbrace{\delta_1 \big( {\color{black}\partial_t \bu} +  \bu \SCAL \GRAD \bu + \GRAD p - {\color{black}2 \nu \Delta \bu}, \
%    \bu \SCAL \GRAD \bv\big)}_{\text{SUPG}}
%    =0, \\
%    \hspace{2.2em} 
%    \big( \DIV \bu,\ q)&
%    + \underbrace{\delta_1 \big( {\color{black}\partial_t \bu} + \bu \SCAL \GRAD \bu + \GRAD p - {\color{black}2 \nu \Delta \bu},\ \GRAD q \big)}_{\text{PSPG}} = 0 
%  \end{aligned}
%\end{equation}



%\newpage
%\bibliographystyle{siam}
%\bibliographystyle{abbrvnat} 
%\bibliographystyle{elsarticle-num}
%\bibliography{ref}
\end{document}

\bibliographystyle{siam}
\bibliography{ref}
\end{document}
